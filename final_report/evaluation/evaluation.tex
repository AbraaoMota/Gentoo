
\chapter{Evaluation}
	

\section{Intended functionality}
In order to be able to perform the functionality as proposed above, the project should be able to demonstrate the following characteristics:

\begin{enumerate}
	\item \textbf{Recommendation system} - The browser extension must be able to passively analyse web traffic and produce visual cues that indicate to the user that it has detected a potential vulnerability. From these, the user should be able to initiate a session where they can try and detect a vulnerability.
	
	\item \textbf{Passive mode} - The extension is designed to eventually be able to take active steps to detect a vulnerability. It's undesirable for the extension to do this if these active steps may infringe terms of service of using the application. Active steps in this context encompass any automated features of the extension that behave on behalf of the user, such as sending web requests, filling out forms etc. Passive steps include the extension reading and analysing web traffic explicitly performed by the user, including suggesting recommendations based on the results of this analysis (the extension should not be liable for any user behaviour that may infringe terms of service, so if passive mode is not enabled, the user has no excuses so as to say they were not aware that the tool infringed these terms without their knowledge). Therefore, the extension must have a clearly visible toggle such that the user can enable or disable active vulnerability detection steps. 
	
	\item \textbf{Action Replay} - As described before, the tool should be able to record user inputs in a targeted domain from when they initiate a vulnerability scanning session until the user declares it is finished. The tool should then be able to replicate these steps automatically, while changing and fuzzing inputs when doing so. Each replication should attempt to reset the web application state inbetween tries to accurately represent fuzzing user input from the state they defined as a starting point for that particular scan.
	
	\item \textbf{Educational functionality} - In order to be able to meet its purpose of being educational to users who don't necessarily run the extension with prior web security knowledge, the tool should have a limited library of plaintext knowledge to explain different vulnerability types, how these may be detected, and suggestions on how to exploit them thereafter. Snippets of this knowledge may be presented under the recommendations shown to users when using the extension. A more detailed explanation may be provided in the pop-up page that appears when the user clicks on the extension icon in the browser.
\end{enumerate}


\section{Metrics of success}

To appropriately judge the success of the extension, we must have some quantifiable metrics, described below

\begin{enumerate}
	\item \textbf{Time to vulnerability} - As mentioned in \ref{limitations}, measuring the total time taken to perform a scan in this human driven context is unfair; the tool does not set a boundary on total scans unlike an automated scanner does. Measuring time taken is however a useful metric, so in order to not completely circumvent using time, another more appropriate metric is suggested - \emph{time to vulnerability}. This will be measured as the total number of seconds taken from the moment a scan is initiated (the user clicks the visual cue given by the tool), until the scan is finished. A scan may finish under two circumstances: a vulnerability has been successfully reported or detected (this may happen at the first attempt by the user, or at a later try when the tool has fuzzed some inputs during the 'action replay' phase), or when the tool reports it could not find a vulnerability. These 2 conditions are bounded in terms of possible time taken, allowing the total time to be measured.
	
	\item \textbf{Interaction volume} - A metric that is sometimes found when evaluating automated scanners  is that of bytes sent and received by the application \cite{stateOfArtAutomatedBlackBoxWebAppVulnTesting}. This quantifies the impact the tool has when stressing the website by fuzzing inputs. The smaller this metric is, the more efficiently the tool is performing its job, by detecting vulnerabilities with fewer web requests sent.
	
	\item \textbf{Number of replays needed} - A similar metric to interaction volume, this number measures how many 'replays' are required by the tool to successfully detect a vulnerability. Both of these metrics measure the efficiency of the tool in detecting vulnerabilities, but this metric more accurately tests the efficiency of the fuzzing engine provided by the tool. Ideally, the tool uses the inputs which are known to work with higher success rates first, and the more esoteric fuzzes would come last, when there is already a decreased likelihood of them working anyway. It also encourages the tool to only include meaningful and relevant fuzzing techniques per vulnerability type, otherwise the tool \emph{could} theoretically fuzz forever. The fewer number of replays needed by the tool in order to find a vulnerability, the better it is performing.
	
%	\item \textbf{Recommendation to vulnerability conversion rate} - An automated scanner is often measured in terms of how many false positives it shows, because it is expected to either positively or negatively declare the existence of a vulnerability. This would be an unfair means of benchmarking this extension - 
\end{enumerate}


\section{Experiments}


\subsection{Test benches}
In order to test my tool during development, it is not necessary to build a vulnerable web application from scratch. Not only would this be time consuming (beyond the purposes of building a tool to exploit such an application), it would also bias the ensuing development of the extension so that it is tailored to successfully detect every injected vulnerability in the web application. Fortunately, there already exist tools dedicated to this purpose. \\

As previously mentioned, DVWA is a vulnerable web application written in PHP, using MySQL for database interactions \cite{dvwaSite}. This application contains a good selection of predefined vulnerabilities to test against:
\begin{multicols}{3}
\begin{itemize}
\item Brute Force Login
\item Command Execution
\item CSRF
\item File Inclusion
\item SQL Injection
\item Upload Vulnerability
\item XSS	
\end{itemize}
\end{multicols}

Another existing tool is WebGoat \cite{webgoatIntro, webgoatGithub}, supported by OWASP. This is an actively supported project by the open source community, and also contains a healthy amount of potential vulnerabilities for developers to test against:

\begin{multicols}{3}
\begin{itemize}
\item 	Access Control Flaws
\item 	AJAX Security
\item 	Authentication Flaws
\item 	Buffer Overflows
\item 	Code Quality
\item 	Concurrency
\item 	Cross-Site Scripting (XSS)
\item 	Improper Error Handling
\item 	Injection Flaws
\item 	Denial of Service
\item 	Insecure Communication
\item 	Insecure Configuration
\item 	Insecure Storage
\item 	Malicious Execution
\item 	Parameter Tampering
\item 	Session Management Flaws
\item 	Web Services
\item 	Admin Functions
\end{itemize}
\end{multicols}

Similarly to the above, there exist more testing tools of this kind, such as WackoPicko \cite{wackoPickoGithub} and HacmeBank \cite{hacmeBankMcAfee}. A more extensive list of existing applications of this kind has been produced by OWASP \cite{owaspVulnerableWebAppsList}. For the purposes of the experiments in this project, I will test against DVWA, WebGoat and WackoPicko, in order, as needed per vulnerability. This provides a good variety of implementations of vulnerabilities to test against; if the tool successfully finds the targeted vulnerabilities in these applications then it is likely to do well in other contexts.

\subsection{Test methodology}
The main method of testing my extension is to pit my extension against other existing automated vulnerability scanners. Some potential candidate scanners include \emph{OWASP ZAP} \cite{owaspZapPage}, \emph{w3af} \cite{w3af} and \emph{burpsuite} \cite{burpSuitePage}. All the scanners will analyse the aforementioned vulnerable applications used in testing. To avoid the case where the produced extension overfits its scanning model to the test vulnerable applications, this experiment will include more applications from the OWASP vulnerable web apps list \cite{owaspVulnerableWebAppsList}, not previously used in testing. \\

Since the browser extension makes use of user input, it is important to test the application with people of different security backgrounds. This relies on adequate classification in this group by the person undertaking the test. There will be 3 proposed categories of user experience when testing the tool; advanced, intermediate and beginner. An advanced user is expected to be well versed in web security, and have had prior experience in diagnosing (perhaps exploiting) vulnerabilities. An intermediate user may be someone getting to grips with this area, perhaps a student who is only now learning about these concepts, but hasn't \emph{necessarily} got experience in detecting or exploiting vulnerabilities. Beginner users are expected to be web savvy, people who are acquainted with using browsers and web applications, but are not necessarily interested or knowledgeable in web security. It is hoped that enough data is gathered to be able to have at least 5 unique people per suggested user group - it may be particularly hard to find advanced users that are willing to test the tool, whereas users who fit the other categories should be much easier to find. \\

To quantify how well the tool performs its job, the success rates of all 3 groups will be analysed when using the extension to find vulnerabilities. A very successful implementation of the project will have made it easy for non-experts to detect vulnerabilities, meaning that results from the beginner group would not vary very much from those in the intermediate and advanced groups using the tool. Therefore, inter-group vulnerability detection success rates will be analysed. From these success rates, it may be possible to extrapolate data on how educational the project was to users in the beginner and intermediate groups. This data could be further backed up by an additional quiz on whether the user has understood the type of vulnerability they detected, and whether they understand the ramifications of doing so by providing an example of a potential exploit they might design as a result. \\

Additionally, comparing each group success rate to the success rates of each of the automated scanners is useful to be able to validate the claim that a user driven, semi-automated approach is advantageous. This should be done within the context of what vulnerabilities the extension is able to detect - if an automated scanner can find more types of vulnerabilities than the ones the extension has been designed to find, then these will not be counted in the results. For a fair comparison in that aspect, it is assumed that with more development time, the extension may be developed further to be able to identify another type of vulnerability. \\

Another means of testing the success of the application would be to put it to test against real world applications. By activating the described passive mode as needed, and browsing web application domains, it is possible that a user of the extension finds vulnerabilities in the target application. Should this happen, it would be a great validating factor for the success of the tool.
