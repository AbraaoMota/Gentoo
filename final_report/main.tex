
\documentclass[a4paper, twoside]{report}

%% Language and font encodings
\usepackage[english]{babel}
\usepackage[utf8x]{inputenc}
\usepackage[T1]{fontenc}
\usepackage{array}
%% Sets page size and margins
\usepackage[a4paper,top=3cm,bottom=2cm,left=3cm,right=3cm,marginparwidth=1.75cm]{geometry}

\usepackage[table, x11names,dvipsnames, table]{xcolor}
%% Useful packages
\usepackage{amsmath}
\usepackage{graphicx}
\usepackage[colorinlistoftodos]{todonotes}
\usepackage[colorlinks=true, allcolors=blue]{hyperref}
\usepackage{url}
\usepackage{multicol}
\usepackage{graphicx}
\usepackage{caption}
\usepackage{subcaption}
\setcounter{secnumdepth}{3}

\usepackage{fancyhdr} % page layout

\usepackage{listings}
\usepackage{color}
\usepackage{float}

\definecolor{codegreen}{rgb}{0,0.6,0}
\definecolor{codegray}{rgb}{0.5,0.5,0.5}
\definecolor{codepurple}{rgb}{0.58,0,0.82}
\definecolor{backcolour}{rgb}{0.95,0.95,0.92}

\lstdefinestyle{mystyle}{
	backgroundcolor=\color{backcolour},   
	commentstyle=\color{codegreen},
	keywordstyle=\color{magenta},
	numberstyle=\tiny\color{codegray},
	stringstyle=\color{codepurple},
	basicstyle=\footnotesize,
	breakatwhitespace=false,         
	breaklines=true,                 
	captionpos=b,                    
	keepspaces=true,                 
	numbers=left,                    
	numbersep=5pt,                  
	showspaces=false,                
	showstringspaces=false,
	showtabs=false,                  
	tabsize=2
}

\lstset{style=mystyle}



%\setlength{\arrayrulewidth}{1mm}


\renewcommand{\footrulewidth}{0.75pt}
\renewcommand{\headrulewidth}{0.75pt}

\pagestyle{fancy}

\pagenumbering{roman}

\title{Gentoo - Webapp vulnerability detection through semi-automated black-box scanning}
\author{Abra\~{a}o Pacheco Dos Santos Peres Mota}
% Update supervisor and other title stuff in title/title.tex

\begin{document}
\input{title/title.tex}

\clearpage{\pagestyle{fancy}\cleardoublepage}
\begin{abstract}
In a world where cyber security is a growing concern in every Internet connected business and service, there is now an explosion of web application users like never before. In using these services, often times users render away valuable data about themselves - only to have it stolen in cyber attacks. It is often the case that in these attacks the vulnerability which facilitated it could have easily been fixed, but not enough emphasis is placed on web security to take preventative measures against this. \\

In the effort against this we find web application scanners, tools designed to analyse websites to detect vulnerabilities so that these may be found and removed. Many of the tools of this type out there are entirely automated, but how feasible or useful would a semi-automated, human centered tool be? \\

We develop Gentoo, a Google Chrome extension which is designed with this purpose in mind. We benchmark it against other well known web application scanners in scanning purposefully vulnerable web applications. We determine that this approach to the problem fills a particular niche that has not yet been explored by many automatic tools, leveraging the position of power that a browser extension has. We also show how Gentoo was used in finding and exploiting a real vulnerability on a live website, validating its use in a practical scenario.
\end{abstract}


\clearpage{\pagestyle{fancy}\cleardoublepage}

\renewcommand{\abstractname}{Acknowledgements}
\begin{abstract}
	
I would like to thank Dr Sergio Maffeis for both his teaching and supervision both before and throughout this project, his help was invaluable in completing this project. \\

I want to thank Dom, Giulio, Mike and Elias for their friendship and patient help throughout my time at Imperial, and for giving me good laughs inbetween. I also want to thank Diogo for his unwavering support in the years I've known him. \\

I want to thank my brothers and sisters in Clapham, who have given me a home outside of home, and become my extended family. \\

\vspace{25mm}
Mas acima de tudo quero agradecer \`a minha familia porque sempre estiveram e sempre estar\~ao comigo. Amo-vos a todos. Sempre.
\end{abstract}

\tableofcontents
%\listoffigures
%\listoftables

\pagenumbering{arabic}
\chapter{Introduction} 

\section{Motivation}

In recent years, use of internet applications has skyrocketed across the world. This is exarcebated by the ubiquity and sheer number of devices that are now connected to the internet. 
The users of these devices place a great deal of trust in the applications and websites they use to power the activities they engage in. 
These play an ever increasingly influential role in people's lives - as an example, in a not so distant past, people would have to travel to a physical bank branch to deal with account matters or execute transactions. Though physical presence can still be required nowadays, a majority of the population will now take advantage of online banking in their day to day lives, often even from the comfort of the phone in their pocket. This has obvious time and efficiency advantages, and benefit many who use this today.
This was not an overnight phenomenon however; online banking initially faced heavy customer reluctance, enough to warrant studies on the cause of this \cite{KUISMA200775}. 
This is understandable, given that everyone holds their financial situation as a very sensitive part of their lives.
Banking is not a solitary example however; with the pervasive use of the internet, its users have gradually become less apprehensive about surrendering important details over a web connection, such as their phone numbers, addresses or medical history.  
This sensitive information is expected to be kept private when being given away to a web service - it is a conventional unspoken agreement between the user and the service provider that this is the case, although there are laws to enforce this as well.
All of this builds up to a massive responsibility placed on the shoulders of web application developers today; their products are expected to uphold a high standard of security, which oftentimes is hard to reach and maintain. \\
% cite some news articles here

Despite this, it is all too common to hear about web applications that suffer from severe cyber attacks. In some of the most debilitating hacks we have heard of in recent years, it is often the case that they were a result of simple vulnerability mitigation measures not being taken \cite{stuxnextHacks, wannaCryHacks, equifaxHacks}. A vulnerability in this context can be illustrated as an unlocked window into a house - it may not be immediately clear that the house owner has overlooked this security aspect, but if a burglar manages to work out that this is the case, they can maliciously \textit{exploit} this vulnerability in the house, and use that as an entry point to steal all the valuable contents within. Properly closing and locking all the windows in the house would be an obvious mitigation to this, but this is only one of the potential (ingenious) ways for a burglar to make his way into the house. The same principle can be applied to websites; it is important to cover as many bases as possible to prevent a potential information or content leak to malicious users. Some vulnerabilities may be harder to detect than others, and it is unlikely that \textit{all} the possible vulnerabilities will be covered, but any efforts towards mitigating any vulnerabilities can only work in favour of the website developer.  \\

Sadly, due to the immaturity of the web development industry and how quickly technologies emerge in the field, web security is an often overlooked aspect of development. 
The lack of security as a fundamental tenet for development is also due to a gap in developer education, and a high entry barrier to understand and mitigate potential security vulnerabilities \cite{veracodeDevopsSurvey}. 
Though this view is slowly changing, web security is not treated as an important focus for new web developers to understand as part of many online and university courses, so many will get by, even work in professional development roles without having ascertained basic security principles. \\

Common vulnerability mitigation measures are also hidden in the inner workings of many frameworks developers use and become accustomed to. 
For example, using \textit{Anti CSRF} (Cross Site Request Forgery) tokens in web forms to prevent action hijacking has become commonplace in web frameworks, and is a feature that is often activated by default \cite{djangoCSRF, rubyOnRailsCSRF, laravelCSRF}.
It is very often the case that features like this will be used without knowledge of what they do and how they work (in my own experience, I had been venturing in web development for years before realising that feature existed). 
However, in the case that the developer changes to use another framework without \textit{secure by default} features, or decides to write an application from scratch, the burden of creating a secure application lies with them even further. 
These default features become like training wheels for some uneducated developers and without them, these creators will be left without a clue as to how to mitigate vulnerabilities, let alone know that these exist altogether. 
Experienced web developers are less likely to leave the "windows" of their website unlocked, but it nevertheless makes sense to install security alarms to prevent both obvious and more subtle security risks.
If these security systems can automatically work in the background it is ideal, but like a home security alarm, it is no good if no-one intervenes upon detection of a problem. 
This begs the question; what is the feasibility of creating a tool that can work as an aide to a web developer in detecting and preventing vulnerabilities in a website? \\

\section{Objectives}
The question raised above effectively highlights the overarching problem this project aims to tackle - improving security for web applications. The goal is to do this through a pragmatic approach; the final objective of this project is to construct a tool which can diagnose vulnerabilities on a target website as a user browses the service.  \\

My initial proposal in solving this encompasses writing a browser extension to analyse the target website, and applying a wide variety of scans and techniques to detect potential security pitfalls the website may have.
A browser extension is an appropriate approach to this as it is a lightweight application running with elevated privileges on the user's browser, giving it the appropriate clearance level to run a variety of automated scans on the user's behalf.  \\


A tool of this kind has 2 immediately clear use cases. 
The first would be to give this tool to a website owner or developer and create it in such a way so that it gives clear and concise suggestions to quantifiably improve the security level of the target website - for example, if an SQL injection has been detected, show the user where on the website this vulnerability can be found, and make effective suggestions as to how to mitigate this (in an SQLi, it may be done by sanitizing user input). The tool could analyse a range of potential vulnerabilities and generate a quantifiable rating for the website in order to give feedback to the developer on where the website needs improving or immediate work.
The slightly alternative use case of this tool provides a more in-depth scan per potential vulnerability, and would be better suited for use by a penetration tester, or an otherwise knowledgeable web security expert. In this use case, the tool would work in a similar fashion, albeit with a different final goal of going 'all the way' by helping the user to detect vulnerabilities and using these to forge exploits on the target application. \\ 
 
In both potential use cases, the benefits of the tool are twofold - it can be used as an educational stepping stone for developers to further their understanding of security in web applications. It also provides a pragmatic way to improve website security, albeit through different routes. In the first case, the developer applies the given suggestions to their website, making an immediate effect on its security. Alternatively, the penetration tester can  show developers the dangers of ignoring security for their website by exploiting open vulnerabilities, which gives further incentive to fix these as soon as possible. A penetration tester with the knowledge of exploiting vulnerabilities will most often also know how to mitigate vulnerabilities against their own attacks. \\


This project aims to explore the latter of the two approaches mentioned. 
This is the more encompassing of the two possible use cases, and produces richer information beyond safety improvement recommendations. 
The extension will be designed to run in real time as the penetration tester navigates the website being targeted, analysing server responses from the website given to the tester based on different inputs. 
It will also perform automated fuzzing of detected user input forms and other parameters in an attempt to detect vulnerabilities that can be triggered from tampering or manipulating these.
It is not reasonable for the penetration tester to find a combination of inputs that will immediately result in the unveiling of a vulnerability; this is a process that requires careful crafting of inputs that adapt to received server output.
It makes sense to take advantage of computing power in this case to expedite this process by attempting several combinations and permutations of inputs that are known to cause problems in insecure systems, unveiling the existence of vulnerabilities.
%The creation of the tool will have quality of vulnerability detection as a more important design principle over breadth of detection. This results in a more accurate tool that achieves better results, with fewer false-positives.

%This choice is made with the evaluation of the tool in mind. False positives (declaring that a website has a vulnerability when in fact it doesn't) are a major evaluation metric for the project and minimising these through more accurate, in-depth scans will result in a quantifiably better, more usable tool. \\  


\section{Contributions}

%
In order to be able to perform the functionality as proposed above, the project should be able to demonstrate the following characteristics:

\begin{enumerate}
	\item \textbf{Recommendation system} - The browser extension must be able to passively analyse web traffic and produce visual cues that indicate to the user that it has detected a potential vulnerability. From these, the user should be able to initiate a session where they can try and detect a vulnerability.
	
	\item \textbf{Passive mode} - The extension is designed to eventually be able to take active steps to detect a vulnerability. It's undesirable for the extension to do this if these active steps may infringe terms of service of using the application. Active steps in this context encompass any automated features of the extension that behave on behalf of the user, such as sending web requests, filling out forms etc. Passive steps include the extension reading and analysing web traffic explicitly performed by the user, including suggesting recommendations based on the results of this analysis (the extension should not be liable for any user behaviour that may infringe terms of service, so if passive mode is not enabled, the user has no excuses so as to say they were not aware that the tool infringed these terms without their knowledge). Therefore, the extension must have a clearly visible toggle such that the user can enable or disable active vulnerability detection steps. 
	
	\item \textbf{Action Replay} - The tool should be able to record user inputs in a targeted domain from when they initiate a vulnerability scanning session until the user declares it is finished. The tool should then be able to replicate these steps automatically, while changing and fuzzing inputs when doing so. Each replication should attempt to reset the web application state inbetween tries to accurately represent fuzzing user input from the state they defined as a starting point for that particular scan.
	
	\item \textbf{Educational functionality} - In order to be able to meet its purpose of being educational to users who don't necessarily run the extension with prior web security knowledge, the tool should have a limited library of plaintext knowledge to explain different vulnerability types, how these may be detected, and suggestions on how to exploit them thereafter. Snippets of this knowledge may be presented under the recommendations shown to users when using the extension. A more detailed explanation may be provided in the pop-up page that appears when the user clicks on the extension icon in the browser.
\end{enumerate}





\chapter{Background}
Types of scanning - network based vs host based scanning


\section {Website Vulnerabilities}

This project is rooted in identifying and mitigating web application vulnerabilities. In order to do so, it is essential to understand what these are, their impact, and what steps are necessary to prevent them. Fortunately, there are great wealths of information available to learn more about vulnerabilities. A great starting point is \textit{OWASP}, the Open Web Application Security Project. This is a community driven effort into improving the safety of software across the world, and the organisation has taken extensive steps into creating useful guides for developers wishing to know more. A particularly convenient resource they provide is a consensus of the top 10 security risks that web applications face today. \\

At the time of writing, this list contains the following risks, some of which are appropriate to pursue in this project: \\

	\emph{Injection} - This risk arises from any place on a website that accepts client controlled input. This may be through the more obvious - submission of web forms and search boxes, but also includes more subtle ways of the user providing input, such as URL or HTTP request parameters. Accepting this input per se is not a vulnerability, but the issue lies in blindly trusting this input to not be malicious. Whenever this input is used to query a database or perform server-side commands, if it has not been sanitized (cleaned to delete potentially dangerous input combinations), it has potential to leak or permanently corrupt information for the website owner. Due to its prevalence and modus operandi, this is an appropriate vulnerability to scan for and detect in this project. \\
	
	\emph{Broken Authentication} - This can encompass a wide range of things, such as use of weak or default passwords and admin accounts, or poor management of session identifiers such that these can be easily manipulated. It can also include flawed password recovery mechanisms. These risks could be analysed as part of the security analysis ran by the tool, and with support from user input could be combined with scans to effectively detect weak authentication mechanisms. \\
	
	\emph{Sensitive Data Exposure} - This risk is a result of using weak or poorly protected cryptographic measures. If a website has left their encryption keys in plaintext for someone to be able to find, or doesn't use HTTPS altogether, it may be exposed to this attack vector. The tool in question could look for a lack of TLS enforcement across pages, or attempt to force a connection downgrade (from HTTPS to HTTP) to guide the user in the right direction of finding sensitive information. \\
	
	\emph{XML External Entities} - This vulnerability exists in applications that accept or include XML data from a 3rd party. A malicious user could use this data format to attempt to exfiltrate sensitive data from the handling server. Identifying this risk without user input may prove to be difficult, but could be within the potential vulnerabilities considered by the tool. \\
	
	\emph{Broken Access Control} - This risk is comprised of all the possible ways in which an application might allow a user to perform actions that should be restricted to their access level (for example, allowing a non-admin user to read bank balances of arbitrary accounts in the system). Determining what is and isn't an allowed action on a website varies tremendously per application, so this is a very difficult task to automate, and isn't ideal to try and incorporate as part of the final tool.   \\
	
	\emph{Security Misconfiguration} - A poorly setup server may suffer from this risk if there are components or services installed by default that are not prepared accordingly to the necessary security measures - such as disabling error stack traces from services; these may reveal more than what a website owner thinks when in the context of being attacked. This vulnerability type is well suited for automatic scans that scour the website for versioning details of services being ran, which may in turn reveal known weaknesses to look for in the case of a negligent set up.\\
	
	\emph{Cross-Site Scripting (XSS)} - One of the most well known risks for web developers today is XSS - this is exploited when a user successfully injects Javascript into a website, causing a non-intended script to run. This is a severe risk depending on how many users it might affect, it can range from Self XSS which affects only the user injecting this content, but can also be seen as Stored XSS, whereby a malicious script is stored in a database, and can be potentially retrieved and ran by other users, posing serious risks where credentials and other session information can be stolen. This lends itself well to the purpose of the application to be developed in this project, as it can test a wide variety of known inputs to expose this vulnerability. \\
	
	\emph{Insecure Deserialisation} - Serialisation is the process of converting a data structure into a format that can be easily transferred over a connection. This resulting new format must then be deserialised to obtain the initial information back. An attacker could craft a serialized object such that it exploits the process or properties of deserialisation in the target application to obtain access to privileged data. This process will vary depending on the application domain and intended data structures, so it is not ideal for automated tools to attempt to tackle this issue. \\ 
	
	\emph{Using components with known vulnerabilities} - In application architectures that heavily rely on a variety of components or libraries from different sources, it can often be hard to ensure that these are all kept up to date. In instances where they are not, it becomes a simple task for a scanner to produce a map of outdated version numbers to possible exploits that have been discovered on the component since then. Once a CVE (Common vulnerabilities and Exposures - a unique identification method for security vulnerabilities found worldwide) is produced as a result from a scan, it is just a matter of reproducing exploit steps found online to endanger the application. \\ 
	
	\emph{Insufficient Logging and Monitoring} - An ideal web application keeps a track of all activities and accesses that occur. This helps provide accountability for actions. When this is not the case, it weakens the position of the website administrators to pinpoint attack culprits. This is very hard to detect from an outsider perspective, and \textit{insufficient} is an objective term; it wouldn't be fruitful to include this in an externally ran vulnerability scanner. \\


\section{Vulnerability Scanners}

\subsection{Black Box}

\subsection{Automated scanning}
This project is not the first piece of work dedicated to solving this problem using 

 
 
 Human judgement is needed anyway - scanner won't know for sure. It's a human process to hack, and people will spend hours in crafting specialised attacks. 
\section{Browser extensions}
 
 
 
 
 

 
\section{Limitations}

\subsection{Human review}

 \subsection{Breadth of  work}
 
 \subsection{Self security}
 	 My own extension is not free from attacks / bugs
http://slideplayer.com/slide/4352044/

\subsection{Ethics \& Handling of Results}

scans might take the website down if intensive
 

\chapter{Design}


\section{Project Contribution}
\label{contribution}

We thus arrive at the intended innovation behind the project idea - to create a semi-automated web application vulnerability scanner. The essential difference between this and the aforementioned black box scanners is that it intends to work as an aide alongside a pentester, not as a fully automated background process, leveraging its role as a Chrome Extension. \\

The proposition of creating a tool that is driven by user input seems appealing for a number of reasons:

\begin{itemize}
	\item \textbf{Full Experience} - A scanner that analyses a web application for vulnerabilities as a user is using the application has fewer limitations as to what is within its scope. As pointed out by \cite{whyJohnnyCantPentest}, scanners struggle to scour through more complex constructs of the internet such as convoluted Javascript forms, AJAX requests and Flash objects. As a result, many scanners ignore these features altogether, dismissing potential vulnerabilities in the process. A user driven experience allows the user to interact as they would normally with these technologies, and can analyse inputs and outputs accordingly.
	
	\item \textbf{Educational} - As mentioned in the introduction, one of the issues that leads to the need of vulnerability scanners is the lack of security-aware developers. By developing a tool that works alongside developers, they can refine their skills in this area if they already know some security basics. There is also room for learning for website owners as penetration testers may demonstrate the severity of exploits caused by vulnerabilities using the tool, and mitigating these. 
	
	\item \textbf{Simpler crawling module} - Since the tool is not scouring the entire site at once but is rather following the more natural workflow of a human user, the equivalent crawling module will only need to keep a much smaller representation of the website as opposed to before. A proposed methodology to simplify this process given the project specification is to create a crawler based on recommendations and an 'action replay' mechanism. 
	
	The recommendation algorithm will need to be based off of similar existing algorithms in automatic crawlers. Since the tool must be driven by user input, the crawler would analyse contents and interactions of the web application with the user (as it would do automatically), and suggest specific features to the user as a starting point for the scan. This can be done by passively reading the contents of web requests between the user and the web application and flagging up any that seem to exhibit behaviour of an insecure system, such as passing user inputs in clear text, or using client-side inputs to control important application sensitive logic. Once a user chooses to follow a recommendation, they may investigate the flagged feature more closely.
	
	At this point, the 'action replay' algorithm begins - once a user has elected a potential target to test for a vulnerability, the tool begins to record user actions. Depending on the selected feature and what vulnerabilities may be discoverable, the tool can then suggest a stopping point of recording, or wait for the user to determine when their actions that detect a vulnerability are enough. The tool then takes these actions and analyses their outputs - it may be the case that a user has found a vulnerability on their first go. It could also be the case that this input didn't trigger a vulnerability, but is worth looking into further. At this point, the crawler will begin to fuzz different versions of input that may be more effective at showcasing vulnerabilities. In recent research by Parvez et al. into evaluating black box scanners \cite{analysisOfEffectivenessOfBlackBoxWebAppScannersStoredSQLStoredXSS}, one of the final recommendations for a better scanner was to add interactive multistep options to the scanner, which is a main focus of this method. To the best of my knowledge, this 'action replay' algorithm is a novel approach in this area.
	
	\begin{figure}[h]
		\centering
		\includegraphics[width=\textwidth]{images/action_replay.png}
		\caption{A visualization of the proposed action replay algorithm. The tool records user input for a time period determined by the user. The tool then replicates actions using fuzzed inputs to try and uncover vulnerabilities if the first attempt was unsuccessful.}
		\label{fig:test}
	\end{figure}
	
	
	\item \textbf{Human judgement} - The previously mentioned paper by Parvez et al. also mentions that choosing the right attack vectors to exploit vulnerabilities is still a big challenge for black box scanners \cite{analysisOfEffectivenessOfBlackBoxWebAppScannersStoredSQLStoredXSS}. Huang et al. recently built a scanner that performed well against open-source competition \cite{webAppSecThreatsCountermeasuresPitfalls}. However, these authors recently acknowledge that \emph{"there is no silver bullet for web application security; threats will continue to grow and evolve"}. Herein lies much of the motivation for making this project user-centric: vulnerability detection and even exploitation may be automated \cite{darpaAIChallenge}, but for now, efforts in having to build an AI that is able to do this are astronomical - it takes years for teams of experts to build tools that do this. Following its recent success, the winner of the DARPA Cyber Grand Challenge (an AI-only capture the flag contest) was pitted against human professionals at the 2016 DEF CON CTF Challenge, where it came last \cite{defcon16Results}. These efforts show a bright future for fully automated security systems, but these may still be some years away in the full making. This project does not admit defeat in attempting to automate the process of countering malicious agents, but rather aims to increase the likelihood of doing a good job against them by covering our bases with higher quality base defences. In fact, even limited domain knowledge has shown to be useful for human penetration testers; a group of students with 'average' security skills achieved a higher success rate on their own than some vulnerability scanners in analysing web applications \cite{whyJohnnyCantPentest}. It is hoped that with the correct suggestions provided by the tool, this project lowers the minimum requirements of a successful user to be 'below average' in their security domain knowledge. \\
	
\end{itemize}



\section{Limitations}
\label{limitations}

In order to be able to effectively evaluate this project at a later point, it is important to delineate realistic expectations in terms of what it can and cannot do. 

\subsection{Time taken}
Since the extension is user driven and not fully automatic, the suggested process will be inevitably slower than if it was otherwise automated - there will be waiting times as users will not instantly attempt to uncover recommended vulnerabilities. There may also be some intellectual effort involved in crafting initial input to try and do this, which adds to this. Both of these factors add to 2 different metrics: the total scan time and the time taken from recommendation to decision of whether a vulnerability is found or not. \\

Total scan time may be especially hard to measure, and can be seen as a disadvantage of this scan. It is not known whether a user will eventually go through every possible page that is relevant, so it becomes very difficult to claim an end to the scan. On the contrary, automated crawlers \emph{will eventually} come to an end their search, and thus put an upper bound on how many resources they can analyse. Human driven crawling may stop or resume at any given point. Not being able to best assess the total scan time is an acceptable tradeoff of this project as it will simply not be avoidable given human driven interaction is being explored. \\

The time taken to ascertain whether a vulnerability exists or not from a recommendation can however be bounded. This period begins when a user decides to investigate a recommendation on their own or by suggestion of the tool, and finishes either when enough proof of a vulnerability has been observed, or when all possible fuzzing opportunities in exploiting it have been exhausted - the extension will have a list of possible fuzzing mutations per type of vulnerability, so this work is bounded by that list. Again, since this is not a fully automated tool, the extension is expected to take longer in this metric due to human interaction when comparing it to automated scanners, which is only natural. \\

\subsection{Breadth of  work}
Section \ref{vulnerabilities} describes many potential kinds of vulnerabilities that the project may choose to tackle. Some of these are harder to detect than others, and thus require more development work. This large scope may make it tempting to try and undertake too many things at once. At the time of writing, it is also hard to assess just how difficult it may be to implement scanning for a specific type of vulnerability. The extension will be developed with aims of finding revelant vulnerabilities - sorted by both volume of occurrence and contemporary relevance as found on surveys. One of the most recurring vulnerabilities in both aspects is Cross Site Scripting, so developing the tool to be able to detect this class of weaknesses is a good starting point. Other features that may seem small (such as analysing cookies or the use of iFrames on a page), may also be beneficial to implement. As the project develops, it becomes harder to weigh the costs of effort to implement versus the success rate of focusing on a specific feature, so some further time should be allocated to allow for this meta research. \\ 

\subsection{Self security}
The security of the extension being developed should not be taken for granted. As mentioned above, Google Chrome has several mechanisms in place to safeguard extensions from falling prey to malicious attackers. Namely, these are:
\begin{itemize}
	\item \textit{Isolated Worlds}
	\item Privilege separation
	\item Predefined permissions
	\item CSP (Content Security Policy)
\end{itemize} 
Although these practices make it much easier for a developer to avoid serious mistakes, it is still possible to write vulnerable extensions. The \textit{threat model} in this case (the way we choose to archetype our potential enemy), is by means of a web attacker. This would be someone who sets up a 'honeypot' website, expecting the extension to scan it but in the process attempt to compromise the extension by different means. Due to the priveleges granted to an extension, this may result in the jeopardizing of sensitive user data, such as their passwords. A recent paper by Carlini, Felt and Wagner reinforces the notion that Chrome's existing techniques are effective in preventing extension compromise, but also list some developer practices that could result in vulnerabilities, such as the unrestricted use of the \texttt{eval} function by Javascript (which is known to be dangerous as it executes given strings as commands), and injecting website data into HTML \cite{evalChromeExtensionSecurityArchitecture}. Many of the notions mentioned in the paper are also cited in Google's own documentation on how to write Chrome extensions \cite{chromeExtensionArchitecture}. Following these best practices will decrease potential risks associated with this project. \\ 

\subsection{Ethics \& Handling of Results}
\label{ethics}
This project aims to build a tool that helps users find vulnerabilities in web applications. Obviously, this may raise ethical ramifications as to how the tool interacts with websites, and how its output is handled. \\

As the tool needs to send requests on the user behalf when executing (especially so during the 'action replay' phase described above), the rate at which this is done may be of concern. Conventional automated scanners may fire 100's of different fuzz attempts at a web application, \emph{per vulnerability, per existing page} - a shotgun-like approach. This approach is very intensive, and for web applications set up for smaller amounts of traffic, may result in a Denial of Service (DoS). This tool aims to reduce this by only passively reading user crawled content, and only actively interacting with the website whenever a flagged vulnerability has been detected. \\

A related concern is to do with the scope of testing and uses on a real web application. For testing and evaluation purposes, the tool will be ran against existing, known to be vulnerable applications, such as DVWA \cite{dvwaSite}. Ideally, it should also be ran against other web applications beyond my control. However, for some of these, it may be the case that running the unhinged extension may infringe usage terms and agreements. For this purpose, a proposed restrained mode for the tool may be built, such that it does not actively take any action when browsing web applications, but rather only passively reads network traffic to deduce and recommend potential vulnerabilities. \\

Naturally, as per the project ideals, any vulnerabilities that may be found as a result of running this extension will duly be reported to the appropriate developers.
\chapter{Implementation}


%5\section{Extension Structure}

As explained in \ref{browserExtensions}, the extension is split into different files for both security reasons and other separation of concerns.
The diagram below puts into more detail the exact structure of the extension in context with some of the real files being used.


\begin{figure}[h]
	\centering
	\includegraphics[width=0.9\textwidth]{images/project_structure.png}
	\caption{The directory structure used for the project}
	\label{fig:test}
\end{figure}

I have arranged the files in question into 2 separate subfolders - the \texttt{testpages} directory keeps the source code for the test harness page with vulnerabilities, while the \texttt{src} directory keeps all the other extension related code. \\

Within \texttt{src} we find different folders for separate concerns - \texttt{lib} stores any library code I have imported from a third-party, \texttt{css} keeps custom styles used across the extension, \texttt{img} is where any images are kept, \texttt{js} holds any custom produced Javascript files, and is where most of the logic within the extension lives. \texttt{src} also contains the \textit{manifest} file, and any HTML page code. 

\section{Manifest}

The manifest is where the extension declares its intentions by enumerating all the scripts and files to be accessed, as well as the permissions required to run the extension as an add-on to the browser. This file is necessary due to security reasons; each extension is expected to run as a standalone program within the browser. Any dependencies are to be declared and included within the package before the program is run, with the exception of contents that are whitelisted in the declared \textit{Content Security Policy} (CSP) directive within the file. I am using the recommended default CSP directive of \texttt{script-src 'self'; object-src 'self'}. This policy actively prevents notoriously dangerous Javascript functions from being evaluated, disables in-line Javascript functionality (which enforces a separation of content from behaviour), and, as the name suggests, will only load scripts and files locally available to the package \cite{chromeExtensionCSP}. \\ 

A potential methodology to use in a setting like this would be to employ a bundler to create a single minified (or compressed) \texttt{.js} file that includes all the required libraries and code to be imported. An example of this is  \textit{Webpack} \footnote{\url{https://webpack.js.org/}} - I did not employ a bundler like this as I learned about its uses later into the project. Furthermore, I am including a relatively small number of third-party sourced code, making this a small practical concern. \\

Other noteworthy details from the manifest file include:

\begin{itemize}
	\item The \texttt{devtools\_page} directive - this is necessary to access the developer tools API's within the extension, which provide extra information when using the extension such as the contents of the requests sent at any given time. This also provides a potential point of extension to the UI of the Chrome developer tools when inspecting pages.
	
	\item The \texttt{web\_accessible\_resources} directive allows me to specify resources which should be accessible in the context of a normal browsing experience. This is a feature enabled in the more recent 2.0 version of the manifest, which blocks resources by default unless they are whitelisted in this manner. This prevents malicious attacks on the extension, such as fingerprinting or exploiting XSS vulnerabilities \cite{chromeExtensionWebAccessibleResources}. An example of a resource included here is the \texttt{request\_logger.html} page.
	
	\item The \texttt{content\_scripts} to be included in every page are also defined here - this includes both CSS files as well as third-party libraries and Javascript vulnerability scanning files.
\end{itemize}

\section{Test harness}

In order to be able to appropriately test for features in development for this extension it is necessary to have access to a fragile website. It would not be ethical to use a website in production to test my extension against; doing so could result in all manner of adversities for the owner of that web address. That is under the assumption that I could firstly find a fragile website to test against; despite the prevalence of security weaknesses spread through the web, it is inherently a difficult task to identify and exploit these. Therefore, I have developed a deliberately weak website to test against. As will be demonstrated in each of the following sections, this harness website (stored under \texttt{testpages/index.html}) can be used to showcase an example of each vulnerability and consequent attack. \\

In order to emulate the website as a server, I am using the out-of-the-box implementation of Python's \texttt{SimpleHTTPServer}. To run this, I simply have to run \texttt{python3 -m http.server 8000} to start a server on the localhost at port 8000. This allows me to submit forms and thus emulate query parameter submissions.

\section{Popup page}

This is the main interaction point for a user utilizing the extension. The layout of this page makes a clear distinction between the different features available to use in the extension. \\

\begin{figure}[h]
	\centering
	\includegraphics[width=0.8\textwidth]{images/first_look.png}
	\caption{The main contents of the popup page when first clicked by a user.}
	\label{fig:test}
\end{figure}

\begin{figure}[h]
	\centering
	\includegraphics[width=0.8\textwidth]{images/popup_2.png}
	\caption{The main contents of the popup page when first clicked by a userw}
	\label{fig:test}
\end{figure}

This page offers an accessible means for tweaking and toggling all the options available in the extension, as well as viewing and analysing the outputs from the different modes of running the extension.  


\section{Background Page}

The background page I am using in the extension is a persistent page, meaning it is constantly running. This is in contrast to event background pages, which are opened and closed as necessary depending on the occasion. I am using this page to listen for messages sent from other scripts to be able to set aesthetic updates, as well as establishing a connection with the \texttt{devtools} page to subscribe to and forward custom request messages to content scripts.

\section{Message Design}



\section{Action Replay}



\section{Passive Mode}

Mention about 2 different ways of doing it - passive mode with normal cross checks by default - sliding window with eviction of requests.
Instead I keep window size and cross check only when enabled.

\subsection{Recommendations}
\chapter{Applications}

This section provides practical examples through which this extension shows itself to be useful and relevant in fulfilling its purpose of detecting vulnerabilities in websites. Firstly, we will demonstrate how the \textit{Recommendations} feature can be used as a low effort entry point into analysing websites for potential vulnerabilities. Secondly, we will use our Test Harness to demonstrate the uses of the \textit{Action Replay} feature in tweaking and replaying innocent requests to produce behaviour that exposes vulnerabilities on a website. We then show how the \textit{Passive Mode} behaves by showcasing how it performs against the intentionally vulnerable test page. Each of these sections will begin with a preamble covering how the example we will go through is vulnerable to establish a better understanding on how the extension exposes said vulnerability. Finally, we run through a case study of a live website vulnerability that we discovered and exploited as a result of using the extension on it.

\section{Recommendations}

In order to set up the example to demonstrate a use case of the \textit{Recommendations}, we set up the Test Harness with 2 intentional XSS Vulnerabilities that work in the same way - any input that is submitted in either the \texttt{injection} or \texttt{injection2} input fields is reflected on the page below its respective input (in the \texttt{output} and \texttt{output2} fields respectively). The source code of the page at this point is shown in Listing \ref{lst:test_harness}, and the resulting rendered page is shown in Figure \ref{fig:test_harness_default}. \\


%\begin{minipage}{\linewidth}
\begin{lstlisting}[label={lst:test_harness}, language={HTML}, caption={The HTML contents of the Test Harness for this example. Two of the inputs in the page are reflected onto the page when they are submitted as query parameters (i.e. seen in the page URL)}]
<script>
	window.onload = function() {
		var url = new URL(window.location.href);
		
		var c = url.searchParams.get("injection");
		if (c) {
			document.getElementById("output").innerHTML = c;
		}
		
		var d = url.searchParams.get("injection2");
		if (d) {
			document.getElementById("output2").innerHTML = d;
		}
	}
</script>

<h1>Vulnerability Testing page</h1>

<form id="boop" action="">
	<input type="text" name="injection" value=""/>
	<input type="text" name="unused" value="">
	<input type="submit">
</form>
<h2 id="output"></h2>

<form id="beep" action="">
	<input type="text" name="unused" value=""/>
	<input type="text" name="injection2" value="">
	<input type="submit">
</form>
<h2 id="output2"></h2>
\end{lstlisting}
%\end{minipage}



\begin{figure}[h]
	\centering
	\begin{subfigure}{.45\textwidth}
		\centering
		\includegraphics[width=.8\linewidth]{images/test_harness_recommendations.png}
		\label{fig:test_harness_original}
	\end{subfigure}%
	\begin{subfigure}{.6\textwidth}
		\centering
		\includegraphics[width=.8\linewidth]{images/test_harness_reflection_1.png}
		\label{fig:test_harness_reflection_1}
	\end{subfigure}
	\caption{The default Test harness page to demonstrate XSS vulnerabilities. Submitting the first input causes the reflection in the second image.}
	\label{fig:test_harness_default}
\end{figure} 

Now we enable the recommendations in this page - setting the sensitivity parameter to \textbf{1}. At this stage, one input is added per form on the page, encouraging the user to "investigate" the first input on each of the forms, seen in \ref{fig:test_harness_recommendations_2}. 

\begin{figure}[h]
	\centering
	\includegraphics[width=0.7\textwidth]{images/test_harness_recommendations_2.png}
	\caption{Enabling the recommendations on the Test Harness}
	\label{fig:test_harness_recommendations_2}
\end{figure}

If the user decides to test the first input on the page, one of the attempted attacks will inject the following into the input:

\begin{center}
	\texttt{<img src=a onerror="window.location.replace( 'chrome-extension://ID/request\_logger.html?ref=CURRENT\_URL')">}
\end{center}

where the \texttt{ID} parameter is replaced by the ID given by Chrome to the extension, and the \texttt{CURRENT\_URL} is the current web address before the form was submitted. This is an example of an XSS attack which surreptitiously executes Javascript on a page by injecting an \texttt{<img>} tag with an invalid source location. This triggers an error when the image is later loaded on that page (since we have established that this input is reflected back onto the page). At this point, the \texttt{onerror} callback given to the input is executed, and in our case, will change the location of the current page to our previously mentioned \texttt{request\_logger} page, passing the current URL as a query parameter in this referral. \\


The \texttt{request\_logger} page is equipped to deal with this specific type of request, by alerting the user both through the extension and in the page output that an XSS vulnerability has been found, indicating the address of the culprit page. The only (current) means to reach this page in the extension is through the payloads prepared in the attacks, meaning that Javascript had to have been executed by one of these, thus indicating an XSS vulnerability. Upon opening the extension for further details, the initial Test Harness URL is then listed as a website vulnerable to XSS. \\


\begin{figure}[h]
	\centering
	\includegraphics[width=	\textwidth]{images/request_logger_warning.png}
	\caption{The request logger page warns the user that the reason they've reached this page is likely to have been due to an XSS vulnerability. Note that the extension badge is also updated with a warning \textbf{(!)} label to indicate this (arrow added in diagram for emphasis).}
	\label{fig:request_logger_warning}
\end{figure}

\begin{figure}[h]
	\centering
	\includegraphics[width=	\textwidth]{images/xss_vulnerable_url.png}
	\caption{Opening the extension confirms that the suspicious URL was indeed vulnerable to XSS.}
	\label{fig:xss_vulnerable_url}
\end{figure}












\section{Action Replay}

\section{Passive Mode}


\section{Live Vulnerability Test Case}

http://studios.fitpass.co.in/login
%
\chapter{Experimentation}
	

\section{Intended functionality}
In order to be able to perform the functionality as proposed above, the project should be able to demonstrate the following characteristics:

\begin{enumerate}
	\item \textbf{Recommendation system} - The browser extension must be able to passively analyse web traffic and produce visual cues that indicate to the user that it has detected a potential vulnerability. From these, the user should be able to initiate a session where they can try and detect a vulnerability.
	
	\item \textbf{Passive mode} - The extension is designed to eventually be able to take active steps to detect a vulnerability. It's undesirable for the extension to do this if these active steps may infringe terms of service of using the application. Active steps in this context encompass any automated features of the extension that behave on behalf of the user, such as sending web requests, filling out forms etc. Passive steps include the extension reading and analysing web traffic explicitly performed by the user, including suggesting recommendations based on the results of this analysis (the extension should not be liable for any user behaviour that may infringe terms of service, so if passive mode is not enabled, the user has no excuses so as to say they were not aware that the tool infringed these terms without their knowledge). Therefore, the extension must have a clearly visible toggle such that the user can enable or disable active vulnerability detection steps. 
	
	\item \textbf{Action Replay} - As described before, the tool should be able to record user inputs in a targeted domain from when they initiate a vulnerability scanning session until the user declares it is finished. The tool should then be able to replicate these steps automatically, while changing and fuzzing inputs when doing so. Each replication should attempt to reset the web application state inbetween tries to accurately represent fuzzing user input from the state they defined as a starting point for that particular scan.
	
	\item \textbf{Educational functionality} - In order to be able to meet its purpose of being educational to users who don't necessarily run the extension with prior web security knowledge, the tool should have a limited library of plaintext knowledge to explain different vulnerability types, how these may be detected, and suggestions on how to exploit them thereafter. Snippets of this knowledge may be presented under the recommendations shown to users when using the extension. A more detailed explanation may be provided in the pop-up page that appears when the user clicks on the extension icon in the browser.
\end{enumerate}


\section{Metrics of success}

To appropriately judge the success of the extension, we must have some quantifiable metrics, described below

\begin{enumerate}
	\item \textbf{Time to vulnerability} - As mentioned in \ref{limitations}, measuring the total time taken to perform a scan in this human driven context is unfair; the tool does not set a boundary on total scans unlike an automated scanner does. Measuring time taken is however a useful metric, so in order to not completely circumvent using time, another more appropriate metric is suggested - \emph{time to vulnerability}. This will be measured as the total number of seconds taken from the moment a scan is initiated (the user clicks the visual cue given by the tool), until the scan is finished. A scan may finish under two circumstances: a vulnerability has been successfully reported or detected (this may happen at the first attempt by the user, or at a later try when the tool has fuzzed some inputs during the Action Replay phase), or when the tool reports it could not find a vulnerability. These 2 conditions are bounded in terms of possible time taken, allowing the total time to be measured.
	
	\item \textbf{Interaction volume} - A metric that is sometimes found when evaluating automated scanners  is that of bytes sent and received by the application \cite{stateOfArtAutomatedBlackBoxWebAppVulnTesting}. This quantifies the impact the tool has when stressing the website by fuzzing inputs. The smaller this metric is, the more efficiently the tool is performing its job, by detecting vulnerabilities with fewer web requests sent.
	
	\item \textbf{Number of replays needed} - A similar metric to interaction volume, this number measures how many 'replays' are required by the tool to successfully detect a vulnerability. Both of these metrics measure the efficiency of the tool in detecting vulnerabilities, but this metric more accurately tests the efficiency of the fuzzing engine provided by the tool. Ideally, the tool uses the inputs which are known to work with higher success rates first, and the more esoteric fuzzes would come last, when there is already a decreased likelihood of them working anyway. It also encourages the tool to only include meaningful and relevant fuzzing techniques per vulnerability type, otherwise the tool \emph{could} theoretically fuzz forever. The fewer number of replays needed by the tool in order to find a vulnerability, the better it is performing.
	
%	\item \textbf{Recommendation to vulnerability conversion rate} - An automated scanner is often measured in terms of how many false positives it shows, because it is expected to either positively or negatively declare the existence of a vulnerability. This would be an unfair means of benchmarking this extension - 
\end{enumerate}


\section{Experiments}


\subsection{Test benches}
In order to test my tool during development, it is not necessary to build a vulnerable web application from scratch. Not only would this be time consuming (beyond the purposes of building a tool to exploit such an application), it would also bias the ensuing development of the extension so that it is tailored to successfully detect every injected vulnerability in the web application. Fortunately, there already exist tools dedicated to this purpose. \\

As previously mentioned, DVWA is a vulnerable web application written in PHP, using MySQL for database interactions \cite{dvwaSite}. This application contains a good selection of predefined vulnerabilities to test against:
\begin{multicols}{3}
\begin{itemize}
\item Brute Force Login
\item Command Execution
\item CSRF
\item File Inclusion
\item SQL Injection
\item Upload Vulnerability
\item XSS	
\end{itemize}
\end{multicols}

Another existing tool is WebGoat \cite{webgoatIntro, webgoatGithub}, supported by OWASP. This is an actively supported project by the open source community, and also contains a healthy amount of potential vulnerabilities for developers to test against:

\begin{multicols}{3}
\begin{itemize}
\item 	Access Control Flaws
\item 	AJAX Security
\item 	Authentication Flaws
\item 	Buffer Overflows
\item 	Code Quality
\item 	Concurrency
\item 	XSS
\item 	Improper Error Handling
\item 	Injection Flaws
\item 	Denial of Service
\item 	Insecure Communication
\item 	Insecure Configuration
\item 	Insecure Storage
\item 	Malicious Execution
\item 	Parameter Tampering
\item 	Session Management Flaws
\item 	Web Services
\item 	Admin Functions
\end{itemize}
\end{multicols}

Similarly to the above, there exist more testing tools of this kind, such as WackoPicko \cite{wackoPickoGithub} and HacmeBank \cite{hacmeBankMcAfee}. A more extensive list of existing applications of this kind has been produced by OWASP \cite{owaspVulnerableWebAppsList}. For the purposes of the experiments in this project, I will test against DVWA, WebGoat and WackoPicko, in order, as needed per vulnerability. This provides a good variety of implementations of vulnerabilities to test against; if the tool successfully finds the targeted vulnerabilities in these applications then it is likely to do well in other contexts.

\subsection{Test methodology}
The main method of testing my extension is to pit my extension against other existing automated vulnerability scanners. Some potential candidate scanners include \emph{OWASP ZAP} \cite{owaspZapPage}, \emph{w3af} \cite{w3af} and \emph{burpsuite} \cite{burpSuitePage}. All the scanners will analyse the aforementioned vulnerable applications used in testing. To avoid the case where the produced extension overfits its scanning model to the test vulnerable applications, this experiment will include more applications from the OWASP vulnerable web apps list \cite{owaspVulnerableWebAppsList}, not previously used in testing. \\

Since the browser extension makes use of user input, it is important to test the application with people of different security backgrounds. This relies on adequate classification in this group by the person undertaking the test. There will be 3 proposed categories of user experience when testing the tool; advanced, intermediate and beginner. An advanced user is expected to be well versed in web security, and have had prior experience in diagnosing (perhaps exploiting) vulnerabilities. An intermediate user may be someone getting to grips with this area, perhaps a student who is only now learning about these concepts, but hasn't \emph{necessarily} got experience in detecting or exploiting vulnerabilities. Beginner users are expected to be web savvy, people who are acquainted with using browsers and web applications, but are not necessarily interested or knowledgeable in web security. It is hoped that enough data is gathered to be able to have at least 5 unique people per suggested user group - it may be particularly hard to find advanced users that are willing to test the tool, whereas users who fit the other categories should be much easier to find. \\

To quantify how well the tool performs its job, the success rates of all 3 groups will be analysed when using the extension to find vulnerabilities. A very successful implementation of the project will have made it easy for non-experts to detect vulnerabilities, meaning that results from the beginner group would not vary very much from those in the intermediate and advanced groups using the tool. Therefore, inter-group vulnerability detection success rates will be analysed. From these success rates, it may be possible to extrapolate data on how educational the project was to users in the beginner and intermediate groups. This data could be further backed up by an additional quiz on whether the user has understood the type of vulnerability they detected, and whether they understand the ramifications of doing so by providing an example of a potential exploit they might design as a result. \\

Additionally, comparing each group success rate to the success rates of each of the automated scanners is useful to be able to validate the claim that a user driven, semi-automated approach is advantageous. This should be done within the context of what vulnerabilities the extension is able to detect - if an automated scanner can find more types of vulnerabilities than the ones the extension has been designed to find, then these will not be counted in the results. For a fair comparison in that aspect, it is assumed that with more development time, the extension may be developed further to be able to identify another type of vulnerability. \\

Another means of testing the success of the application would be to put it to test against real world applications. By activating the described passive mode as needed, and browsing web application domains, it is possible that a user of the extension finds vulnerabilities in the target application. Should this happen, it would be a great validating factor for the success of the tool.


\chapter{Evaluation}

Project evaluation is very important, so it's important to think now about how you plan to measure success. For example, what functionality do you need to demonstrate?  What experiments to you need to undertake and what outcome(s) would constitute success?  What benchmarks should you use? How has your project extended the state of the art?  How do you measure qualitative aspects, such as ease of use?  These are the sort of questions that your project evaluation should address; this section should outline your plan.


\section{Intended functionality}

\section{Metrics of success}

\subsection{Correct reporting}

\subsection{False positives}

\subsection{Code coverage}

\subsection{Usability}

\section{Experiments}

Running an experiment where the volume of lifting done by the extension allows me to find a high number of vulns

Time taken

volume of interaction with the server

Accuracy of recommendations (not exactly false positives because its not fair to do so)

true positives - 


Passive mode of the extension to throw at a real website .. more muzzled - passive analysis of the traffic to try and flag up things that are interesting and 'tamperable' 
\subsection{Benchmarks}

\subsection{Ease of use}

\subsection{Educational purpose}

\section{Intellectual contributions}

%\input{future/future.tex}
\chapter{Conclusion}

We conclude this thesis with some ideas on how to further improve the body of work thus far, and make an overall collection of thoughts on Gentoo. \\

\section{Future work}

It is easy to see how Gentoo could be taken some steps further. One of the first potential additions to this would be to incorporate the scanning for SQL injetions. This particular type of vulnerability would express itself in a very similar way to what has already been shown in the Reflected XSS. The concept of these two vulnerabilities is similar - a malicious input is injected into a form, and the output is analysed in search of clues of injection. In the case of encoding an SQL Injection analyser, this might involve searching the response body for database-like query outputs, or for error pages that slowly leak information such as the version of the database, or react to blind injections (perhaps the page takes longer to load because a successful delay command has been executed as part of the malicious payload). Given the framework we have built up with Gentoo, detecting and preventing SQL injections would be the next logical step to follow in expanding the suite of vulnerabilities Gentoo can detect. \\

Another improvement would be to make more effective use of XMLHttpRequests, particularly in the Action Replay mode. In the current code version, Gentoo will load a new attack window per generated attack - doing so allows the extension to evaluate any Javascript ran as part of a potential attack in the new window, since the evaluated body is not part of the initial request response. Doing so can become extremely cumbersome however, especially if we were to expand the suite of attacks. At the moment, the automatic launch of several new tabs can severely slow down the machine, or in more memory-starved situations, crash Google Chrome altogether. Using XMLHttpRequests would bypass this situation; these requests are performed in the background, and require minimal to no input from the user when compared with new tabs being opened. \\

One of the major improvements in attack power would be to enable attack fuzzing. In the current version of Gentoo, each existing attack payload is hardcoded, and none of the attacks will deviate from this. However, with the vast expanse of the internet, it is very likely that a potential website \texttt{A} suffers from a vulnerability when injected with the specific payload \texttt{P}, whereas website \texttt{B} might suffer from an injection of payload \texttt{P'}. It is not the case however that \texttt{A} is vulnerable to payload \texttt{P'}, or \texttt{B} to \texttt{P}
 for that matter. But what if the only difference between \texttt{P} and \texttt{P'} is a single \texttt{"} character? It is hard to justify the hardcoding of all these possibilities, which is where the power of input fuzzing would come in. Using this, only one base input type would have to be hardcoded, whilst all the other inputs would be automatically generated as per the defined regex rules. Using a fuzzer engine to generate attack payloads would exponentially increase the attack surface area of Gentoo, although care would have to be taken to discerningly choose which regex rules would be selected so as to generate good quality input fuzzes. \\
 
 Gentoo could also be improved from a user centric design overhaul. Specifically, an interactive tutorial that guides each new user in a step by step experience that teaches how to use the different attack modes would be of great use. Such a tutorial would be of particularly great aid to new users coming to grips with how the Action Replay mechanism works, as this mode is complex, and the steps to follow and the order in which to execute them is not immediately clear. Adding this feature alongside a text knowledge base explaining different vulnerabilities and their significance would also address the concern of educational deficiency of Gentoo.  Doing so would also give users another chance to become more educated on how to diagnose vulnerabilities. This would improve usability and create a far more polished end product. \\
 
 This project would also have very likely benefitted from benchmarking Gentoo against a more extensive list of competitors. Both ZAP and w3af are open source web application scanners, so we did not ascertain how Gentoo would have performed when set side by side against commercial tools. This leaves a gap for further investigation in the project. \\



\section{Final Remarks}


This project has ambitiously saddled on the margin of automation when it comes to black box web application scanners. Gentoo finds itself as a hybrid tool that automates the bulk of work, but only when it is told to by a user. We think that this hybrid approach is one worth exploring further, especially because certain vulnerability diagnosis and exploitation is a complex task, which is likely to require human validation for a while still. \\

We saw how this automation can be useful in a low effort scenario using the Recommendations, how it was used in active user-led scans by using the Action Replay mechanism, and how it can be leveraged in its entirity for Passive scans. Each of these modes has their own perks and separate advantages. We were particularly content to see the validation of the hypothesis that vulnerability diagnosis and exploit is a delicate process by uncovering the live vulnerability; although this was triggered through use of the Recommendations feature, it was only fully exploited afterwards using trial and error. Doing so legitimized much of the work being done on this project. \\

We think this project explored through its analysis that there is no right or wrong approach in whether a scanner is fully automated or perhaps semi automated, as there are good and valid arguments both for and against each of these variations. Through numerical analysis of the two types, we came to the conclusion that Gentoo's implementation was successful in delivering a focused, lightweight scan, which successfully met some of the key metrics we were hoping to observe from such a tool. Perhaps more importantly here is ascertaining that the abstraction of using a browser extension was particularly advantageous from a user's perspective because the extension leverages its power and granted permissions to co-exist very closely between the user and the application under test. \\

 This takes us back to the initial project proposal - is it possible to design a browser extension which passively analyses the web traffic the user makes and reports it back if signs of a web vulnerability have been detected? Given the analysis performed, and the real-life experience under its belt in such a short period, it's fair to say that Gentoo met most of the project criterion. Although it underperformed in its educational aspect, it effectively leveraged its position as an extension to be able to meet the remaining project goals. Gentoo shows that a scanner working as an aide to a human penetration tester is powerful and fit for purpose.
 





































\appendix
\chapter{Appendix}

\section{Exploit Code} \label{exploitCode}

In the following listing, we include the code used to generate the exploit as seen in Section \ref{live_vulnerability}. In the Listing, \texttt{MY\_CONTROLLED\_SITE} is the domain of a website we control, with the purpose of hosting the \texttt{EVIL.dmg} file.

\begin{lstlisting}[label={lst:exploit_code}, language={HTML}]
"></form>
<div style="height:6px; width:100%; clear:both;"></div>
<form>
	<div class="form-group">
		<label class="control-label" for="password">PASSWORD</label>
		<input class="form-control" type="password" name="LoginForm[password]" placeholder="Password" >
	</div>
	<style>
		.myanchor {
			display: block;
			width: 310px;
			height: 40px;
			background: #4E9CAF !important;
			padding: 10px;
			text-align: center;
			border-radius: 5px;
			font-weight: bold;
		}
	</style>
	<div class="form-group btn-container">
		<a href="http://MY_CONTROLLED_SITE/test/EVIL.dmg" class="myanchor" type="submit" ><i class="fa fa-sign-in fa-lg fa-fw"></i>SIGN IN</a>
	</div>
</form></section></body>

\end{lstlisting}

\bibliographystyle{plain}
\bibliography{bibs/sample.bib}

\end{document}